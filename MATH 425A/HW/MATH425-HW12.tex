\documentclass[oneside]{amsart}
\usepackage[left=1.25in,right=1.25in,top=0.75in,bottom=0.75in]{geometry}
\linespread{1.05}
\usepackage{mathtools}
\usepackage{tcolorbox}
\usepackage{euler}
\usepackage{amssymb,latexsym,amsmath,amsthm}
\usepackage{mathrsfs}
\usepackage{xcolor}
\usepackage{graphicx}
\usepackage{hyperref}
\hypersetup{
    colorlinks = true,
    linkbordercolor = {red}
}
\usepackage[all]{xy}
\usepackage[T1]{fontenc}
\usepackage{xstring}
\usepackage{xparse}
\usepackage{mathrsfs}
% \definecolor{brightmaroon}{rgb}{0.76, 0.13, 0.28}
% \usepackage[linktocpage=true,colorlinks=true,hyperindex,citecolor=blue,linkcolor=brightmaroon]{hyperref}
%\usepackage{fullpage}
% \usepackage[a4paper, total={5.5in, 9in}]{geometry}
\usepackage{tikz-cd}
\theoremstyle{definition}
%% this allows for theorems which are not automatically numbered
\newtheorem{defi}{Definition}[section]
\newtheorem{theorem}{Theorem}[section]
\newtheorem{lemma}{Lemma}[section]
\newtheorem{obs}{Observation}
\newtheorem{exercise}{Exercise}[section]
\newtheorem{rem}{Remark}[section]
\newtheorem{construction}{Construction}[section]
\newtheorem{prop}{Proposition}[section]
\newtheorem{coro}{Corollary}[section]
\newtheorem{disc}{Discussion}[section]
\DeclareMathOperator{\spec}{Spec}
\DeclareMathOperator{\im}{im}
\DeclareMathOperator{\obj}{obj}
\DeclareMathOperator{\ext}{Ext}
\DeclareMathOperator{\Lim}{Lim}
\DeclareMathOperator{\Int}{Int}
\DeclareMathOperator{\tor}{Tor}
\DeclareMathOperator{\ann}{ann}
\DeclareMathOperator{\id}{id}
\DeclareMathOperator{\proj}{Proj}
\DeclareMathOperator{\gal}{Gal}
\DeclareMathOperator{\coker}{coker}
\newcommand{\degg}{\textup{deg}}
\newtheorem{ex}{Example}[section]
%% The above lines are for formatting.  In general, you will not want to change these.
%%Commands to make life easier
\newcommand{\RR}{\mathbf R}
\newcommand{\aff}{\mathbf A}
\newcommand{\ff}{\mathbf F}
\usepackage{mathtools}
% \newcommand{\ZZ}{\mathbf Z}
\newcommand{\pring}{k[x_1, \ldots , x_n]}
\newcommand{\polyring}{[x_1, \ldots , x_n]}
\newcommand{\poly}{\sum_{\alpha} a_{\alpha} x^{\alpha}} 
\newcommand{\ZZn}[1]{\ZZ/{#1}\ZZ}
% \newcommand{\QQ}{\mathbf Q}
\newcommand{\rr}{\mathbb R}
\newcommand{\cc}{\mathbb C}
\newcommand{\complex}{\mathbf {C}_\bullet}
\newcommand{\nn}{\mathbb N}
\newcommand{\zz}{\mathbb Z}
\newcommand{\PP}{\mathbb  P}
\newcommand{\cat}{\mathbf{C}}
\newcommand{\ca}{\mathbf}
\newcommand{\zzn}[1]{\zz/{#1}\zz}
\newcommand{\qq}{\mathbb Q}
\newcommand{\calM}{\mathcal M}
\newcommand{\latex}{\LaTeX}
\newcommand{\V}{\mathbf V}
\newcommand{\tex}{\TeX}
\newcommand{\sm}{\setminus} 
\newcommand{\dom}{\text{Dom}}
\newcommand{\lcm}{\text{lcm}}
\DeclareMathOperator{\GL}{GL}
\DeclareMathOperator{\cl}{cl}
\DeclareMathOperator{\Hom}{Hom}
\DeclareMathOperator{\aut}{Aut}
\DeclareMathOperator{\SL}{SL}
\DeclareMathOperator{\inn}{Inn}
\DeclareMathOperator{\card}{card}
\newcommand{\sym}{\text{Sym}}
\newcommand{\ord}{\text{ord}}
\newcommand{\ran}{\text{Ran}}
\newcommand{\pp}{\prime}
\newcommand{\lra}{\longrightarrow} 
\newcommand{\lmt}{\longmapsto} 
\newcommand{\xlra}{\xlongrightarrow} 
\newcommand{\gap}{\; \; \;}
\newcommand{\Mod}[1]{\ (\mathrm{mod}\ #1)}
\newcommand{\p}{\mathfrak{p}} 
\newcommand{\rmod}{\textit{R}-\textbf{Mod}}
\newcommand{\idealP}{\mathfrak{P}}
\newcommand{\ideala}{\mathfrak{a}}
\newcommand{\idealb}{\mathfrak{b}}
\newcommand{\idealA}{\mathfrak{A}}
\newcommand{\idealB}{\mathfrak{B}}
\newcommand{\X}{\mathfrak{X}}
\newcommand{\idealF}{\mathfrak{F}}
\newcommand{\idealm}{\mathfrak{m}}
\newcommand{\s}{\mathcal{S}}
\newcommand{\cha}{\text{char}}
\newcommand{\ccc}{\mathfrak{C}}
\newcommand{\idealM}{\mathfrak{M}}
\tcbuselibrary{listings,theorems}
\usetikzlibrary{decorations.pathmorphing} 
\newcommand{\overbar}[1]{\mkern 1.5mu\overline{\mkern-1.5mu#1\mkern-1.5mu}\mkern 1.5mu}

%Itemize gap:

% \pagecolor{black}
% \color{white}
% Author info

\title{Math 425A HW12, Nov. 22, 6PM}
\date{November 17, 2022 \\ {Department of Mathematics, University of Southern California}}
\address{Department of Mathematics, University of Southern California, 
Los Angeles, CA 90007}
\begin{document}
\maketitle
\setcounter{tocdepth}{4}
\setcounter{secnumdepth}{4}
 \section*{Chapter 7}
\begin{tcolorbox}[colback=black!5!white,colframe=black!75!black,title= Exercise $2.1.$]  For each of the following sequences $(a_n)_{n=1}^\infty$, prove whether the series $ \sum _{n=1}^\infty a_n$ converges or diverges. (If it converges, you do not need to find the limit.)
\begin{itemize}
	\item [(1)] $a_n = \sqrt{n+1} - \sqrt{n}$.
	\item [(2)] $a_n = \frac{\sqrt{n+1} - {\sqrt{n}}}{n}$.
	\item [(3)] $a_n = ( \sqrt[n]{n}-1)^n$.
	\item [(4)] $a_n = \frac{(-1)^n}{\log n}$ for $n \geq 2$ (and $a_1 = 0$).
\end{itemize}
\tcblower 
\begin{proof} 
(1) We make an observation to the series defined by $a_n$, that is, we note that it is a telescoping: 

\begin{align*}
	\sum_{n=1}^M \sqrt{n+1}-\sqrt{n} & = (\sqrt{2}-1) + (\sqrt{3}-\sqrt{2})+(\sqrt{4}-\sqrt{3}) + \cdots + \sqrt{M+1}-\sqrt{M} \\
	&= \sqrt{M+1}-1.
\end{align*}
Then as $M\to \infty$, we have that the sum tends to infinity. Hence we have the series diverges.

(2) For $|a_n|$, we have $\frac{\sqrt{n+1} - {\sqrt{n}}}{n} = \frac{1}{n(\sqrt{n+1} + \sqrt{n})} < \frac{1}{n \sqrt{n}} = \frac{1}{n^\frac{3}{2}}$. Hence $\sum_{n=1}^\infty a_n = \sum_{n=1}^\infty \frac{\sqrt{n+1} - {\sqrt{n}}}{n}$ converges as we know that $\sum_{n=1}^\infty  \frac{1}{n^\frac{3}{2}}$ converges and so we apply the Comparison Test.

(3) We apply the Root Test. 
\begin{align*}
	|a_n|^{\frac{1}{n}} = |( \sqrt[n]{n} -1)^n|^{1/n} = |\sqrt[n]{n}-1|.
\end{align*}
If $\sqrt[n]{n}-1>0$, then $\limsup (\sqrt[n]{n}-1 )= 1-1 -0 <1 $, by Theorem \textcolor{red}{5.1}; similarly, if $\sqrt[n]{n}-1<0$, then 
$ \limsup (1-\sqrt[n]{n})=1-1 = 0 < 1$. Therefore we have that the sequence converges.

(4) Consider $a_k = \frac{1}{\log k}$ for $k \geq 2$ (as $\log 1 = 0$ and $\log 0$ isn't defined). As we have $\log k< \log k+1$, then $\frac{1}{\log k} > \frac{1}{\log k+1}$, and so we have a monotonically decreasing sequence and clearly $\lim _{n \to \infty} \frac{1}{\log k} = 0$. Therefore after applying the Alternating Series Test we get that $ \sum _{n=1}^\infty  \left (\frac{(-1)^n}{\log k} \right ) = \sum_{n=1}^\infty a_n$ converges.
\end{proof}
\end{tcolorbox}


\begin{tcolorbox}[colback=black!5!white,colframe=black!75!black,title= Exercise $2.2.$] Consider the series 
\[
\sum_{n=1}^\infty \frac{1}{1+z^n}.
\] Determine the values of $z \in \rr$ ($z \neq -1$) make the series convergent and which make it divergent. Prove your answers are correct.
\tcblower 
\begin{proof} To start off, for $z = 0$, we have $\sum_{n=1}^\infty \frac {1}{1+0^n} = 1+1+1 \cdots $ which makes it divergent. For $z =1$, $\sum_{n=1}^\infty \frac {1}{2} = \frac{1}{2} + \frac{1}{2} + \cdots $ which makes it divergent. Furthermore, if $|z|<1$, and $z\neq -1$, then $ \lim_{n \to \infty} \frac{1}{1+z^n} = \frac{1}{1+0} = 1 \neq 0$ and so we cannot have convergence. Hence it remains to look at $z \in (-\infty, -1) \cup (1, \infty)$. Fix $z>1$. Then $z^n<z^n+1$ and so $\frac{1}{z^n} > \frac{1}{z^n+1}$. Now $\sum_{n=1}^\infty \frac{1}{z^n}$ converges by the Root Test as $\limsup \left | \left (\frac{1}{z^n} \right ) ^\frac{1}{n}\right| =\limsup | 1/z| =0<1$. Hence we have that $\sum_{n=1}^\infty \frac{1}{z^n+1}$ converges by the Comparison Test for fixed $z>1$. Lastly, fix $z<1$. Then $(\frac{1}{1+z^n})_{n = 1}^\infty$ is monotonically decreasing sequence and $\frac{1}{1+z^n} \to 0$ as $n \to \infty$. By Alternating Series Test we have that $\sum_{n=1}^\infty \frac{(-1)^n}{1+z^n}$ converges and thus makes $\sum_{n=1}^\infty \frac{1}{1+z^n}$ converge with $z<1$.
\end{proof}
\end{tcolorbox}


\begin{tcolorbox}[colback=black!5!white,colframe=black!75!black,title= Exercise $3.1.$] Assume that $ \sum_{n=1}^\infty a_n$ converges absolutely. Prove that $ \sum_{n=1}^\infty \frac{\sqrt{|a_n|}}{n}$ converges. (Hint: Use the inequality $ 2 AB \leq A^2 +B^2$, valid for any real numbers $A$, $B$.)
\tcblower 
\begin{proof} We use the AM-GM inequality as follows. 
\begin{align*}
	\sqrt{a_n \cdot \frac{1}{n^2}} & \leq \frac{a_n + \frac{1}{n^2}} {2} = \frac{a_n}{2} + \frac{\frac{1}{n^2}}{2} = \frac{a_n}{2}+\frac{1}{2n^2} \\
	\implies \sqrt{ \frac{a_n}{n^2}} = \frac{\sqrt{a_n}}{n} & \leq \frac{a_n}{2}+\frac{1}{2n^2}.
\end{align*}
Now $\sum_{n=1}^\infty  \frac{a_n}{2} = \frac{1}{2} \sum_{n=1}^\infty $ converges as we assumed absolute converges of $\sum_{n=1}^\infty a_n$, and so $\sum_{n=1}^\infty \frac{1}{2n^2} =\frac{1}{2} \sum_{n=1}^\infty \frac{1}{n^2}$ converges as this is a $p$-series with $p =2>1$. Hence we have that their sum converges, i.e. $$ \sum_{n=1}^\infty \left (\frac{a_n}{2}+\frac{1}{2n^2} \right )$$ converges, which forces $\sum _{n=1}^\infty \frac{\sqrt{a_n}}{n}$ to converge as well by the Comparison Test and AM-GM. 
\end{proof}

\end{tcolorbox}

\begin{tcolorbox}[colback=black!5!white,colframe=black!75!black,title= Exercise $3.2.$] 
\
\begin{itemize}
	\item [(1)] Assume that $ \sum_{n=1}^\infty a_n$ and $ \sum_{n=1}^\infty b_n$ converge absolutely. Prove that $  \sum_{n=1}^\infty (a_n+b_n)$ converges absolutely as well.
	\item [(2)] Assume that $ \sum_{n=1}^\infty a_n$ converges. Does it follow that $ \sum_{n=1}^\infty a_{2n}$ converges? Give a proof or a counterexample. 
	\item [(3)] Assume that $ \sum_{n=1}^\infty a_n$ converges absolutely. Does it follow that $ \sum_{n=1}^\infty a_{2n}$ converges absolutely? Give a proof or counterexample. 
\end{itemize}
\tcblower 
\begin{proof} (1) As $ \sum_{n=1}^\infty |a_n|$ and $\sum_{n=1}^\infty |b_n|$ both converge, then for all $\epsilon > 0$, there exists $N_1, N_2 \in \nn$ such that $n \geq m \geq N_1,N_2$ gives $ |\sum_{n=k}^m  |a_k|| < \epsilon /2 $ and  $ |\sum_{n=k}^m  |b_k|| < \epsilon /2 $ by Proposition 1.3. And so 
\begin{align*}
	\left| \sum_{n=k}^m |a_k + b_k| \right | \leq \left | \sum _{n=k}^m |a_k| + |b_k| \right | \l & = \left | \sum _{n=k}^m |a_k| + \sum _{n=k}^m |b_k| \right | \\ 
	&\leq \left | \sum _{n=k}^m |a_k|\right | + \left | \sum _{n=k}^m |b_k|\right | < \epsilon/2+\epsilon/2 = \epsilon, 
\end{align*}

(2) We give a counter example. The series $ \sum_{n=1}^\infty \frac{(-1)^n}{n}$ is convergent where $a_n = \frac{(-1)^n}{n}$, but $a_{2n} = \frac{(-1)^{2n}}{2n} = \frac{1}{2n}$ and so $\sum_{n=1}^\infty a_{2n} = \sum_{n=1}^\infty \frac{1}{2n} = \frac{1}{2} \sum_{n=1}^\infty \frac{1}{n}$ which is the harmonic series scaled by $1/2$ and hence this series is divergent. 

(3) We claim that this statements holds true. Let $\sum_{n=1}^\infty a_n$ be absolutely convergent. Then we have $(|a_n|)$ being a sequence of nonnegative real numbers and now $\sum_{n=1}^\infty |a_{2n}| \leq \sum_{n=1}^\infty |a_n|$ which makes $\sum_{n=1}^\infty |a_{2n}|$ converge absolutely by the Comparison Test.


\end{proof}

\end{tcolorbox}

\begin{tcolorbox}[colback=black!5!white,colframe=black!75!black,title= Exercise $4.1.$] Let $B = \{ 0 \} \cup \{ \frac{-1}{n^2} \}_{n \in \nn}$ and $E = \rr \setminus B$. Consider the series 
\[
\sum_{n=1}^\infty \frac{1}{1+n^2x}
\] on the set $E$.
\begin{itemize}
	\item [(1)] Prove that the series converges absolutely for all $x \in E$; therefore it converges pointwise to a function $f \colon E \to \rr$.
	\item [(2)] Prove that the series converges uniformly to $f$ on $[-\infty , -\delta ] \cup [\delta, \infty) \setminus B$ for any $\delta > 0$, but that it does not converge uniformly to $f$ on $E$.
	\item [(3)] Prove that $f$ is continuous on $E$.
	\item [(4)] Prove that $f(0+) = +\infty$, and therefore $f$ is not a bounded function.
\end{itemize}
\tcblower 
\begin{proof} (1) For $x > 0$ in $E$, we have 
$\sum_{n=1}^\infty \left|\frac{1}{1+n^2x}\right| = \sum_{n=1}^\infty \frac{1}{1+nx^2} \leq \sum_{n=1}^\infty \frac{n^2x}{1} = \frac{1}{x} \sum_{n=1}^\infty \frac{1}{n^2}$ and so this is a $p$-series with $p=2>1$, and therefore we have that $\sum_{n=1}^\infty \frac{1}{1+nx^2}$ converges absolutely by the Comparison Test. Let $x<0$ in $E$. Note that after sufficiently large $n \geq N$, we have $1+nx^2 <-n^2$; we pick $N$ to be the minimum natural number such that $N^2 \geq \frac{1}{|1+x|}$. For $n \geq N$, this gives us $\left | \frac{1}{1+n^2x} \right | \leq \frac{1}{n^2}$. Hence as this is once again the same $p$-series as before with $p=2$, we have that $\sum_{n=N}^\infty \left | \frac{1}{1+n^2x} \right |$ converges and so our original series $\sum_{n=1}^\infty \frac{1}{1+n^2x}$ converges as well as we did so with finitely many terms up to $N$. 

(2) We show this by the Weierstrass $M$-Test. Let $\delta > 0$. We're going to be assuming that $x$ is not in $B$ in the following argument.

 Take $x \in [\delta, \infty) $ and $\ell \in (0, \delta )$. Then $1+n^2x > n^2x > n^2 \ell$, and thus $\frac{1}{1+n^2x} < \frac{1}{n^2 \ell }$. As $\sum_{n=1}^\infty \frac{1}{n^2 \ell } = \frac{1}{\ell }\sum_{n=1}^\infty \frac{1}{n^2}$ converges, then $f$ converges uniformly by Weierstrass $M$-Test. Now for $x \in [-\infty, -\delta]$ we've established at the end of part $(1)$ that gives some $N \in \nn$ such that for $m \geq N$, we have $\left | \sum_{n=m}^\infty  \frac{1}{1+n^2x} \right | \leq \sum_{n=m}^\infty \frac{1}{n^2} $ and this last series converges so we uniform convergence as well. The reason that $f$ doesn't converge uniformly on all of $E = \rr \setminus B$ is because, for example, it does not converge uniformly on $(0, \delta ]$ with $\delta > 0$ as if it did by the Cauchy Criterion we have that there is some $N \in \nn$ such that $ \sum_{n=N}^m \frac{1}{1+n^2x} < \frac{1}{2}$ for all $x \in (0, \delta]$. But this doesn't work as we could then choose $x = \frac{1}{N^2}$, which gives a contradiction. 
 
 (3) The function $f$ is continuous where it it is uniformly continuous on by the Uniform Limit Theorem. As shown in $(1)$, $f(x)$ doesn't converge in $B$. For $b>0$ and $t \in [b, \infty)$ or $a < 0$ and $t \in (-\infty, a]$ we have that $f$ converges uniformly as established in $(2)$. Hence $f$ is continuous at $t$. 
 
 (4) Write $x_k = \frac{1}{k}$, then $\sum_{n=1}^\infty \frac{1}{1+\frac{n^2}{k}} = k \sum_{n=1}^\infty \frac{1}{k+n^2}$ which diverges and hence shows that the function $f$ is not bounded. Alternatively, $f$ cannot be bounded as $f(0) = \sum_{n=1}^\infty \frac{1}{1+n^2(0)} = \sum_{n=1}^\infty 1$ doesn't converge. 
\end{proof}

\end{tcolorbox}


\begin{tcolorbox}[colback=black!5!white,colframe=black!75!black,title= Exercise $4.2.$] Find the radius of convergence for each of the following power series:
\[
\sum_{n=0}^\infty n^n z^n\; \;  \sum _{n=0}^\infty \frac{z^n}{n!}\; \;  \sum_{n=0}^\infty z^n\; \; \sum_{n=1}^\infty \frac{z^n}{n}\; \;  \sum _{n=1}^\infty \frac{z^n}{n^2}.
\] 
\tcblower 
\begin{proof} 

% Recall that this means that we're finding $R = \frac{1}{\alpha}$, where $\alpha  = \limsup \sqrt[n]{|c_n|}$.

For the first power series, $c_n = n^n$ and so $\alpha = \limsup (|n^n|)^\frac{1}{n} = \limsup |n|$, and thus we have that $R = \frac{1}{\alpha} = 0$. Hence the radius of convergence is $R = 0$.

For the second, $c_n = \frac{1}{n!}$, so $\alpha = \limsup \left ( \left |\frac{1}{n!} \right |\right )^\frac{1}{n} = \lim \left ( \left |\frac{1}{n!} \right |\right )^\frac{1}{n} = 0$. Hence the radius of convergence is $R = +\infty$.

For the third, $c_n =1$ for all $n$, and so $ \alpha = \limsup |1|^\frac{1}{n} =1$. Thus the radius of convergence is $R = 1$.

For the fourth, $c_n = \frac{1}{n}$, and so $\alpha = \limsup |\frac{1}{n}|^\frac{1}{n} = \lim |\frac{1}{n}|^\frac{1}{n} = 1$. Therefore the radius of convergence is $R = 1$.

Lastly, for the fifth, $ c_n = \frac{1}{n^2}$ which makes $ \alpha = \limsup | \frac{1}{n^2}|^\frac{1}{n}=\lim |\frac{1}{n^2}|^\frac{1}{n} = \lim \left |\frac{1}{\sqrt[n]{n^2}} \right | = 1$. Hence the radius of convergence is $R = 1$.
\end{proof}
\end{tcolorbox}


\begin{tcolorbox}[colback=black!5!white,colframe=black!75!black,title= Exercise $4.3.$] Consider the power series $ \sum_{n=0}^\infty c_n z^n$. Let $R$ be the radius of convergence of the power series, and assume $R>0$. Let $f \colon (-R, R) \to \rr$ be the function defined by $f(z) = \sum_{n=0}^\infty c_n z^n$. Prove the following statements, which refine Theorem 4.4.
\begin{itemize}
	\item [(1)] For any $r \in (0, R)$, the series $\sum_{n=0}^\infty c_n z^n$ converges uniformly on $(-r, r)$ to $f$.
	\item [(2)] $f$ is continous on all of $(-R, R)$.
\end{itemize}
\tcblower 
\begin{proof} (1) We proceed by applying the Weierstrass $M$-Test. By Theorem 4.4, since $f \colon (-R, R) \to \rr$, i.e. the domain of the function is the radius of convergence of  $\sum_{n=0}^\infty c_n z^n$ and $f \colon z \mapsto \sum_{n=0}^\infty c_n z^n$ then the series converges for all $z$ in the domain of $f$. As $r \in (0,R)$, then it is indeed in the radius of convergence, as $(0, R) \subset (-R, R)$, and furthermore we clearly have that $(-r, r) \subset (-R, R)$. Hence we have convergence of the power series for all $(-r,r)$ to $f$. Now we choose $M_n = (zr)^n= z^n r^n$ for each $f_n$ in the sequence $(f_n)$ as then this gives us $|f_n(z)| \leq M_n$ for each $n$ and therefore by the Weierstrass $M$-Test we have that the series $\sum_{n=0}^\infty c_n z^n$ converges uniformly on $(-r,r)$ as for all $z \in (-r,r)$ we have $z$ being less than $R$ or greater than $-R$ and in either case $\sum_{n=1}^\infty M_n = \sum_{n=1}^\infty |c_nr^n|$ converges as $(-R, R)$ is our radius of convergence.


(2) By $(1)$, we have uniform convergence of $\sum_{n=0}^\infty c_n z^n$ on $(-r, r)$ for an arbitrary $r$ where $0 < r<R$, which means that we have continuity on $(-r,r)$ as we have uniform convergence, and as we have $r \in (0, R)$ for any such $r$ then for $|z|<R$ we get that $f$ is continuous. \end{proof} 

\end{tcolorbox}


\end{document}