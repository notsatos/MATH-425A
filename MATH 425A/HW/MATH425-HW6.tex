\documentclass[10pt,reqno]{amsart}
\usepackage{graphicx}
\usepackage{fullpage}
\usepackage{tcolorbox}
% \usepackage[a4paper, total={5.5in, 8in}]{geometry}
%\usepackage{mathpazo}
%\usepackage{euler}
\usepackage{amsfonts,amssymb,latexsym,amsmath,amsthm}
\usepackage{tikz-cd}
\usepackage{mathrsfs}
\usepackage{stmaryrd}
\usepackage{hyperref}
\hypersetup{
    colorlinks = true,
    linkbordercolor = {red}
}
\theoremstyle{definition}
%% this allows for theorems which are not automatically numbered
\newtheorem{defi}{Definition}[section]
\newtheorem{theorem}{Theorem}[section]
\newtheorem{lemma}{Lemma}[section]
\newtheorem{obs}{Observation}
\newtheorem{exercise}{Exercise}[section]
\newtheorem{rem}{Remark}[section]
\newtheorem{construction}{Construction}[section]
\newtheorem{prop}{Proposition}[section]
\newtheorem{coro}{Corollary}[section]
\newtheorem{disc}{Discussion}[section]
\DeclareMathOperator{\spec}{Spec}
\DeclareMathOperator{\im}{im}
\DeclareMathOperator{\obj}{obj}
\DeclareMathOperator{\ext}{Ext}
\DeclareMathOperator{\Lim}{Lim}
\DeclareMathOperator{\Int}{Int}
\DeclareMathOperator{\tor}{Tor}
\DeclareMathOperator{\ann}{ann}
\DeclareMathOperator{\id}{id}
\DeclareMathOperator{\proj}{Proj}
\DeclareMathOperator{\gal}{Gal}
\DeclareMathOperator{\coker}{coker}
\newcommand{\degg}{\textup{deg}}
\newtheorem{ex}{Example}[section]
%% The above lines are for formatting.  In general, you will not want to change these.
%%Commands to make life easier
\newcommand{\RR}{\mathbf R}
\newcommand{\aff}{\mathbf A}
\newcommand{\ff}{\mathbf F}
\usepackage{mathtools}
% \newcommand{\ZZ}{\mathbf Z}
\newcommand{\pring}{k[x_1, \ldots , x_n]}
\newcommand{\polyring}{[x_1, \ldots , x_n]}
\newcommand{\poly}{\sum_{\alpha} a_{\alpha} x^{\alpha}} 
\newcommand{\ZZn}[1]{\ZZ/{#1}\ZZ}
% \newcommand{\QQ}{\mathbf Q}
\newcommand{\rr}{\mathbb R}
\newcommand{\cc}{\mathbb C}
\newcommand{\complex}{\mathbf {C}_\bullet}
\newcommand{\nn}{\mathbb N}
\newcommand{\zz}{\mathbb Z}
\newcommand{\PP}{\mathbb  P}
\newcommand{\cat}{\mathbf{C}}
\newcommand{\ca}{\mathbf}
\newcommand{\zzn}[1]{\zz/{#1}\zz}
\newcommand{\qq}{\mathbb Q}
\newcommand{\calM}{\mathcal M}
\newcommand{\latex}{\LaTeX}
\newcommand{\V}{\mathbf V}
\newcommand{\tex}{\TeX}
\newcommand{\sm}{\setminus} 
\newcommand{\dom}{\text{Dom}}
\newcommand{\lcm}{\text{lcm}}
\DeclareMathOperator{\GL}{GL}
\DeclareMathOperator{\Hom}{Hom}
\DeclareMathOperator{\aut}{Aut}
\DeclareMathOperator{\SL}{SL}
\DeclareMathOperator{\inn}{Inn}
\DeclareMathOperator{\card}{card}
\newcommand{\sym}{\text{Sym}}
\newcommand{\ord}{\text{ord}}
\newcommand{\ran}{\text{Ran}}
\newcommand{\pp}{\prime}
\newcommand{\lra}{\longrightarrow} 
\newcommand{\lmt}{\longmapsto} 
\newcommand{\xlra}{\xlongrightarrow} 
\newcommand{\gap}{\; \; \;}
\newcommand{\Mod}[1]{\ (\mathrm{mod}\ #1)}
\newcommand{\p}{\mathfrak{p}} 
\newcommand{\rmod}{\textit{R}-\textbf{Mod}}
\newcommand{\idealP}{\mathfrak{P}}
\newcommand{\ideala}{\mathfrak{a}}
\newcommand{\idealb}{\mathfrak{b}}
\newcommand{\idealA}{\mathfrak{A}}
\newcommand{\idealB}{\mathfrak{B}}
\newcommand{\X}{\mathfrak{X}}
\newcommand{\idealF}{\mathfrak{F}}
\newcommand{\idealm}{\mathfrak{m}}
\newcommand{\s}{\mathcal{S}}
\newcommand{\cha}{\text{char}}
\newcommand{\ccc}{\mathfrak{C}}
\newcommand{\idealM}{\mathfrak{M}}
\tcbuselibrary{listings,theorems}
\usetikzlibrary{decorations.pathmorphing} 
\newcommand{\overbar}[1]{\mkern 1.5mu\overline{\mkern-1.5mu#1\mkern-1.5mu}\mkern 1.5mu}

%Itemize gap:

% \pagecolor{black}
% \color{white}
% Author info

\title{Math 425A HW6, Oct. 7, 6PM}
\author{Juan Serratos}
\email{jserrato@usc.edu}
\date{October 3, 2022 \\ {Department of Mathematics, University of Southern California}}
\address{Department of Mathematics, University of Southern California, 
Los Angeles, CA 90007}
\begin{document}
\maketitle
\setcounter{tocdepth}{4}
\setcounter{secnumdepth}{4}

\section{Chapter 3.}

\begin{tcolorbox}[colback=black!5!white,colframe=black!75!black,title= Chapter 3 $\S2.4$: Exercise 2.9.] Suppose $X$ a finite, nonempty, set and suppose $d$ be a metric on $X$. Let $\mathcal T$ denote the topology generated by $d$. Show that $\mathcal T = \mathcal P(X)$. Conclude that any metric on $X$ is equivalent to the discrete metric. (Hint: To show that $\mathcal T = \mathcal P (X)$, start by proving that $\{ x \} = B_{(X,d)} (x, r_x)$ for some sufficiently small $r_x$, for each $x \in X$.)
\tcblower 
\begin{proof}
$(\subseteq)$ As $X$ is finite, then it has finitely many subsets. Recall that the topology on a metric space is the collection of all open sets on the metric space $X$ generated by the corresponding metric. Let $E \in \mathcal T$, i.e. $E$ is an open set in $X$. Then obviously $E$ must be subset of $X$ by definition of an open set, and so $E \in \mathcal P(X)$. 

$(\supseteq)$ Now suppose that $L \in \mathcal P (X)$, i.e. $L \subseteq X$. Then, as $X$ is finite, then there are finitely many points in $L$; we can enumerate $L$ so that $L = \{p_1, p_2, \ldots, p_n \}$. As $X$ is itself open, then $L \subseteq X = \Int_X(X)$. Then for every point $p_i \in L$, $1 \leq i \leq n$, there is some ball $B_X(p_i, r_i)$ with $r_i > 0$ such that $B_X(p_i, r_i) \subseteq X$. Thus it suffices to show that for any $p_i \in L$, \[ L = \bigcup _{i = 1}^n B_X(p_i, r_i).\] Suppose that $q \in L$. Then $q = p_\beta$, where $1 \leq \beta \leq n$, and, moreover, there is some open ball $B_X(p_\beta, r_\beta )$ with $r_\beta > 0$ such that $B_X(p_\beta, r_\beta) \subseteq X$. Thus we see that $p_\beta \in \bigcup_{i =1}^n B_X(p_i, r_i)$. Now suppose that $\ell \in \bigcup_{i =1}^n B_X(p_i, r_i)$. Then $\ell \in B_X(p_\alpha, r_\alpha) \subseteq X$ for some $\alpha$. We claim that $\{p_\alpha \} = B_X(p_\alpha, r_\alpha)$ for some sufficiently small $r_\alpha$. In general, let $x \in X$. Then we take can take $r_x = \min_{y \in X\setminus \{x \}}(d(x,y))$ as $X$ is assumed to be finite. Then $B_X(x, r_x) = \{x \}$ as each open ball of the form $B_X(x, r_x)$, where $r_x = \min_{y \in X} (d(x,y))$, has the property that $t \in B_X(x, r_x)$ if and only if $d(x,t) < \min _{y \in X\setminus \{x \}} (d(x,y))$ and the only point in $X$ that satisfies this strict property with $r_x$ is $x$ itself (and the opposite direction $\{x \} \subseteq B_X(x, r_x)$ is clear). Thus the claim follows, and so $\ell \in \{p_\alpha \}$, i.e. $\ell = p_\alpha \in L$. Hence the equality of sets follows and $L$ is a union of open balls (a posteriori, just a union of singleton open balls) and so $L$ is open in $X$ and thus $L \in \mathcal T$. 

Lastly we can conclude that any metric on a finite set $X$ is equivalent to the discrete metric as our claim in showing the reverse inclusion showed that for any point $x \in X$ there is some $r_x >0 $ which gives $B_X(x,r_x) = \{ x\}$, and as open balls generate the topology in a metric space then any other metric on $X$ will be equivalent to the discrete metric (i.e., the discrete metric is given by our established open balls $B_X(x, r_x) = \{ x\}$) as the open balls $\{x \}$ will be contained in all other hypothetical open balls given by ``another" metric on $X$.
\end{proof}
\end{tcolorbox}
\begin{tcolorbox}[colback=black!5!white,colframe=black!75!black,title= Chapter 3 $\S2.4$: Exercise 2.10.]	 Prove that the Euclidean metric and the square metric are equivalent on $\rr^n$.
\tcblower
\begin{proof} Recall that the usual Eucidean norm in $\rr^n$ is given taken by comparing two points $p = (a_1, \ldots, a_n)$ and $q = (b_1, \ldots, b_n)$ in $\rr^n$ where the metric is given by $d_E(p,q) = \sqrt{(b_1 - a_1)^2 + \cdots + (b_n - a_n)^2}$. And the square norm is given by $d_\infty (p,q ) = \max \{ |b_1 - a_1|, \ldots, |b_n - a_n | \}$ where $p,q \in \rr^n$ again. We claim that $\frac{1}{\sqrt{n}} d_E(p,q) \leq d_\infty(p,q)$. By definition of $d_E$, we have that $d_E(p,q) = \sqrt{(b_1 - a_1)^2 + \cdots + (b_n - a_n)^2 } \leq \sqrt{ \sum _{\ell=1}^n\max \{|b_1-a_1|^2,\ldots, |b_n -a_n |^2\} } = \sqrt{n \max \{|b_1-a_1|^2,\ldots, |b_n -a_n |^2\}} = \sqrt{n} d_\infty (p,q)$, and so $\frac{1}{\sqrt{n}} d_E(p,q) \leq d_\infty (p,q)$ and the claim follows. Furthermore, we have that $d_\infty (p,q) = \max_{1 \leq i \leq n } \{ |b_i -a_i | \} \leq d_E(p,q) = \sqrt{\sum_{i=1}^n (b_i - a_i) }$ since $d_\infty (p,q) = |b_\alpha -a_\alpha|$ for some $\alpha \in \{1, \ldots, n \}$ so $d_\infty^2 (p,q) = (b_\alpha - a_\alpha)^2$ and thus $d_\infty ^2 (p,q) \leq d_E(p,q)^2$ which implies $d_\infty (p,q) \leq d_E(p,q)$. So then if $t \in B_{\rr^n}^E (p, \epsilon)$, where this is the open ball with the Euclidean metric, then $t \in B_{\rr^n}^\infty (p, \epsilon)$, where this is the open ball with respect to the square metric, by the previous sentence; that is, $B_{\rr^n}^E (p, \epsilon) \subseteq B_{\rr^n}^\infty (p, \epsilon)$ for any $p \in \rr^n$ and $\epsilon > 0$. Lastly we claim that $B_{\rr^n}^\infty (p, \frac{\epsilon}{\sqrt{n}}) \subseteq B_{\rr^n}^E (p, \epsilon)$ for any $p \in \rr^n$ and $\epsilon > 0$. Take $s \in B_{\rr^n}^\infty (p, \frac{\epsilon}{\sqrt{n}})$. So then $d_\infty (p, s) < \epsilon/\sqrt{n}$, and by a claim we made earlier, we have that $d_E(p,s) \leq \sqrt{n} d_\infty (p,s) < \epsilon$ and thus $s \in B_{\rr^n}^E (p, \epsilon)$. This establishes our claim and we have that $B_{\rr^n}^\infty (p, \frac{\epsilon}{\sqrt{n}})  \subseteq B_{\rr^n}^E (p, \epsilon)$. The proposition of the Exercise follows.
\end{proof}
\end{tcolorbox}


\begin{tcolorbox}[colback=black!5!white,colframe=black!75!black,title= Chapter 3 $\S3.2$: Exercise 3.1.]  
\begin{tcolorbox}[colback=red!5!white,colframe=red!75!red,title= Proposition 3.4.] 
Let $X$ be a set, and let $\mathcal B$ be a collection of subsets of $X$, which has the following properties: 
\begin{itemize}
	\item [(1)] Every $x \in X$ is contained in at least one element $B$ of $\mathcal B$.
	\item [(2)] If $B_1, B_2 \in \mathcal B$ and $x \in B_1 \cap B_2$, then there exists a $B_3 \in \mathcal B$ such that $x \in B_3 \subseteq B_1 \cap B_2$.
\end{itemize}
Then the following collection $\mathcal T$ is a topology on $X$: 
\[ \mathcal T = \left \{ U \in \mathcal P(X) \colon U = \bigcup _{B \in \mathcal A} B \text{ for some subcollection } \mathcal A \subseteq \mathcal B \right \}.
\]
and $\mathcal B$ is a basis for $\mathcal T$. On the other hand, if $\mathcal T$ is a topology on $X$ and $\mathcal B$ is a subcollection of $\mathcal T$ such that the preceding equation holds, then $\mathcal B$ must satisfy both properties $(1)$ and $(2)$.
\end{tcolorbox}
\begin{proof} 
$(\Rightarrow)$ Firstly, it is easy to see that $\varnothing \in \mathcal T$ since $\varnothing \in \mathcal P(X)$ and $\varnothing$ is in the subcollection $\mathcal A \subseteq \mathcal B$ as it's a collection of sets, so $\varnothing$ is vacuously a union of sets in the subcollection $\mathcal A$. Now $X \in \mathcal P(X)$, and we claim that $X = \bigcup _{W \in \mathcal A} W$ for a subcollection $\mathcal A \subseteq \mathcal B$. If we take $x \in X$, then we have that $x \in B_\alpha$ for some $B_\alpha \in \mathcal B$. So then if we take $\mathcal A$ to be the whole collection $\mathcal B$, then $x \in \bigcup _{W \in \mathcal A} W$ as $x \in B_\alpha \in \mathcal A$ for all $x \in X$. For the reverse inclusion, let $\bigcup_{W \in \mathcal A} W$ be such that $\mathcal A$ has the element $B_\beta \in \mathcal B$ that posses every $x \in X$. Then clearly $\bigcup _{W \in \mathcal A} \subseteq X$. Hence the claim holds and $X \in \mathcal T$. Let $\mathfrak U \subseteq \mathcal T$. We must show that $\bigcup_{U \in \mathfrak U} U  \in \mathcal T$. As each $U \in \mathfrak U \subseteq \mathcal T$, then $U \in \mathcal P(X)$ such that $U = \bigcup_{T \in \mathcal A} T$ for some subcollection $\mathcal A \subseteq \mathcal B$. So then for each $V \in \mathfrak U$ there is a corresponding  $V = \bigcup_{T \in \mathcal A_V} T$, where $\mathcal A_V$ denotes the associated subcollection of $\mathcal B$. Now, for every $U \in \mathfrak U$, then we can denote the union of all corresponding subcollections $\mathcal A_U$ as $\mathfrak A$, i.e. $\mathfrak A = \bigcup _{U \in \mathfrak U} \mathcal A_U$ where $\mathcal A_U$ is a subcollection collection of $\mathcal B$ and $U = \bigcup_{T \in \mathcal A_U} T$ for all $U \in \mathfrak U$. Then $ \bigcup_{U \in \mathfrak U} U = \bigcup \left(\bigcup _{T \in \mathcal A_U} T \right) = \bigcup_{E \in \mathfrak A \subseteq \mathcal B} $
Thus we have that $\bigcup_{U \in \mathfrak U} U \in \mathcal T$ as the preceding rewriting of the union shows and as all $ U \in \mathfrak U \subseteq \mathcal T$ by assumption then $U \in \mathcal P (X)$ (i.e., $U \subseteq X$) so the union of all such $U$ are contained in $X$ once again, that is, in $X$'s power set. Lastly, to show that $\mathcal T$ is indeed a topology we have to show that finite intersections satisfy the corresponding topology axiom. Let $\mathfrak D = \{ V_1, \ldots, V_n \}$ be a finite subset of $\mathcal T$. Now consider $\bigcap_{i =1}^n V_i$, where $V_i \in \mathfrak D$. Then, for all $ 1 \leq i \leq n$, we can write $V_i = \bigcup _{B \in \mathcal A} B$ where $\mathcal A$ is a subcollection of $\mathcal B$; write $\mathcal A_{i}$ for each corresponding subcollection of $\mathcal B$ for $V_i = \bigcup_{B \in \mathcal A_{i}} B$. Then 
\begin{align*}
\bigcap_{i=1}^n V_i = \bigcap_{i =1}^n \left ( \bigcup_{B \in \mathcal A_{i}} B \right),\end{align*}
and we claim that $\mathcal A = \{ B \in \mathcal B \colon \exists B_i \in \mathcal A_i, B \subseteq \bigcap _{i=1}^n B_i \}$ gives a satisfactory subcollection for $\bigcap_{i=1}^n V_i$ to be in $\mathcal T$. That is, we show that $\bigcap_{i=1}^n V_i = \bigcup _{B \in \mathcal A } B$, where $\mathcal A$ is the subcollection we just defined. Let $x \in \bigcap _{i=1}^n V_i$. Then $x \in V_j$ for all $1 \leq j \leq n$. Then $x \in B_1, B_2, \ldots, B_n$ for all $B_j \in \mathcal A_j$ by assumption of $\mathcal A$. So then $x \in B_1 \cap B_2 \cap \cdots \cap B_n$, and so by $(2)$ property, we have that there is some $B_\alpha \in \mathcal B$ such that $x \in B_\alpha \subseteq B_1 \cap B_2 \cap \cdots \cap B_n$, and so $B_\alpha \in \mathcal A$ and $x \in \bigcup _{B \in \mathcal A} B$. For the reverse inclusion, suppose that $\ell \in \bigcup _{B \in \mathcal A} B$. WLOG, let $x \in B$ for some $B \in \mathcal A$. Then $B \subseteq B_1 \cap B_2 \cap \cdots \cap B_n$ for $B_1, B_2, \ldots, B_n \in \mathcal A_i$. Thus $x \in B_i$, for all $1 \leq i \leq n$, so we have that $x \in V_j$ as $\mathcal D =\{V_1, \ldots, V_n \} \subseteq \mathcal T$, and so $x \in \bigcap _{i=1}^n V_i$. Lastly, $\mathcal B$ is a basis for $\mathcal T$ as we have shown that $\mathcal T$ is a topological space, and every element of $\mathcal T$ can be written as a union $\bigcup_{B \in \mathcal A} B$ for some subcollection $\mathcal A \subseteq \mathcal B$, i.e. every element of the topology $\mathcal T$ is written as union of elements from $\mathcal B$


$(\Leftarrow)$ For the opposite direction, suppose that $\mathcal T$ is a topology on $X$ and $\mathcal B \subseteq \mathcal T$ such that $\mathcal T$ is what it is presented as in Proposition 3.4. So as $X \in \mathcal T$, then $X = \bigcup_{Y \in \mathcal A} Y$ for some subcollection $\mathcal A \subseteq \mathcal B$, so if $x \in X$ then $x \in Y_\alpha$ for some $Y_\alpha \in \mathcal A$, but as $\mathcal A \subseteq \mathcal B$, then $x$ is in at least one element of $\mathcal B$. For the second property, let $B_1, B_2 \in \mathcal B$ and $x \in B_1 \cap B_2$. Then, as $\mathcal B \subseteq \mathcal T$, then $B_1 = \bigcup_{Y \in \mathcal Q} Y$ and $B_2 = \bigcup _{T \in \mathcal W} T$, where $\mathcal Q$ and $\mathcal W$ are subcollections of $\mathcal B$. So then $x \in \left ( \bigcup_{Y \in \mathcal Q} Y \right ) \cap \left( \bigcup_{T \in \mathcal W} T \right)$, and so $x \in Y_\alpha$ for some $Y_\alpha \in \mathcal Q$ and $x \in T_\beta$ for some $T_\beta \in \mathcal W$. And $\left ( \bigcup_{Y \in \mathcal Q} Y \right ) \cap \left( \bigcup_{T \in \mathcal W} T \right) = \bigcup_{B \in \mathcal A} B:= E$ where $\mathcal A = \{ B \in \mathcal B \colon B \subseteq Y_\alpha \cap T_\beta, \text{ for } T_\beta \in \mathcal W, Y_\alpha \in \mathcal Q \}$ by the previous work done showing that $\mathcal T$ is a topology. Thus $x \in E$ such that $x \in B$ for some $B \in \mathcal A$ and $B \subseteq Y_\alpha \cap T_\beta$. Hence $x \in B \in \mathcal B$ and $B \subseteq B_1 \cap B_2$.
\end{proof}
\end{tcolorbox}
\newpage 
\begin{tcolorbox}[colback=black!5!white,colframe=black!75!black,title= Chapter 3 $\S3.2$: Exercise 3.2.] Prove that the collection $\mathcal R$ of all \textit{open rectangles} of the form \[ (a_1, b_1) \times (a_2, b_2) \times \cdots \times (a_n, b_n) \; \; \; \; a_j, b_j \in \rr, a_j < b_j, \text{ for all $j$}
\] is a basis for the standard topology on $\rr^n$. 
\tcblower 	
\begin{proof} Let $X = \rr^n$. First we show that $X$ is a union of so-called open rectangles. For all $x \in X$, where $x= (x_1, \ldots, x_n)$, consider $\mathfrak R_X (x, \epsilon) := (x_1-\epsilon, x_1 + \epsilon) \times \cdots \times (x_n - \epsilon, x_n + \epsilon)$, where $\epsilon > 0$. Then we claim that $\bigcup _{x \in X} \mathfrak R_X (x, 1) = X$. Firstly, it is clear that $\mathfrak R_X(x, 1) \subseteq X$, and so we have that $\bigcup _{x \in X} \mathfrak R_X (x, 1) \subseteq X$. Now suppose we have some $x=(x_1, \ldots ,x_n) \in X$. Then $x \in \mathfrak R (x, 1)$ as for all $x_i$ where $1 \leq i \leq n$, we have that $x_i -1 < x_i < x_i + 1$. But then $x \in \mathfrak R_X(x, 1) \subseteq \bigcup_{x \in X} \mathfrak R (x, \epsilon)$. Moreover, we've only used open rectangles with $\epsilon = 1$, and so we have that 
\[
\rr^n = \bigcup_{x \in X} \mathfrak R_X(x, 1) \subseteq \bigcup_{x \in X, \epsilon>0} \mathfrak R_X(x,\epsilon),
\] and so we have the equality holds in equation above as $\mathfrak R_X(x, \epsilon) \subseteq \rr^n$ for any $\epsilon>0$ and $x \in \rr$. Therefore the claim holds and we have satisfied the $(1)$ property needed to use Proposition 3.4. Moving on, suppose we have $R_1, R_2 \in \mathcal R$ where $R_1 = (a_1, b_1) \times \cdots \times (a_n, b_n)$ and $R_2 = (c_1, d_1) \times \cdots \times (c_n, d_n)$ such that for each $1 \leq i \leq n$ we have $a_i < b_i$ and $c_i < d_i$. Let $x \in R_1 \cap R_2$. Then $x= (x_1, \ldots, x_n)$ implies that $x_i \in (a_i, b_i)$ and $x_i \in (c_i,d_i)$ for all $i$ so we thus construct  $R_3 = ((a_1, b_1) \cap (c_1, d_1))\times \cdots \times ((a_n, b_n) \cap (c_n, d_n))$ for which $x \in R_3 \subseteq R_1 \cap R_2$. Therefore we have that $(2)$ is satisfied and can conclude that $\mathcal R$ is basis for some topology on $X$. Now let $F \in \mathcal R$; say, $F = (a_1, b_1) \times \cdots \times (a_n, b_n)$. Then we want to show that $F = \Int_X (F)$ with respect to the Euclidean metric. We know that all open intervals $x \in (a_i, b_i)$ are open with respect to the Euclidean metric, and so there is some $\epsilon_i >0$ for all $i$ such that $x_i\in (x-\epsilon_i, x+ \epsilon_i) \subseteq (a_i, b_i)$, and so following this argumentation we get that $(x_1, \ldots, x_n) \in (x_1 - \epsilon, x_1 + \epsilon_1) \times (x_2-\epsilon, x_2 + \epsilon_2) \times \cdots \times (x_n- \epsilon, x_n + \epsilon) \subseteq F$. Hence all points of $F$ are interior points of $F$ with respect to $X$ (the opposite inclusion is immediate by definition). Furthermore, consider $x = (x_1, \ldots, x_n) \in \rr^n$ and suppose $r > 0$. We claim that $B_X(x, r)$ can be written as a union of elements of $\mathcal R$. As explained Exercise 2.10, we have that the square metric $d^\infty$ (which has ``open balls" as open rectangles as described in Example 2.3. of pages $45-46$ in the Course Notes) and $d_E$ are equivalent on $\rr^n$, and so in particular $d_E (p,q) \leq \sqrt{n} d_\infty (p,q)$ for any points $p,q \in X$, and so $\mathfrak R_X(x, \frac{r}{\sqrt{n}}) \subseteq B_X(x,r)$ which implies that $\bigcup_{x \in X} \mathfrak R_X(x,\frac{r}{\sqrt{n}}) \subseteq B_X(x,r)$. Using Exercise 2.10 again, we get an equivalence in the last line so that we can write $\bigcup_{x \in X} \mathfrak R_X(x,\frac{r}{\sqrt{n}}) = B_X(x,r)$; that is, the we can write an open ball $B_X(x,r)$ as a union of elements of $\mathcal R$, so the claim follows. Therefore, by Proposition 3.6, we must have that $\mathcal R$ is in fact the basis for the standard topology on $\rr^n$ as $\mathcal T = \mathcal R$, where $\mathcal T$ denotes the standard topology on $\rr^n$. 
\end{proof}
\end{tcolorbox}


\section{Chapter 4.}

\begin{tcolorbox}[colback=black!5!white,colframe=black!75!black,title= Chapter 4: Exercise 1.1.] Let $E_1, E_2$ be subsets of a metric space $(X,d)$. Prove that
\[
\Lim_X(E_1 \cup E_2) = \Lim_X(E_1) \cup \Lim_X(E_2).
\]
\tcblower
\begin{proof} $(\subseteq)$ Let $x \in \Lim_X(E_1 \cup E_2)$. Then $x \in X$ and for any open neighborhood $x \in W \subseteq X$, we have that $W \cap ( (E_1 \cup E_2) \setminus \{x \} )$ is nonempty. Now take $\ell \in W \cap ( (E_1 \cup E_2) \setminus \{x \} )$. So then $\ell \in W$ and $\ell \in (E_1 \cup E_2) \setminus \{x \}$, i.e. $\ell \in E_1 \cup E_2$ and $\ell \neq x$. WLOG, suppose that $\ell \in E_1$. Then $\ell \in E_1 \setminus \{x \}$, and so as $\ell \in W$ as well we get that $\ell \in W \cap E_1 \setminus \{x \}$; that is, $W$ intersects with $E_1 \setminus \{x \}$. Hence $x \in \Lim_X(E_1)$.

$(\supseteq)$ Let $x \in \Lim_X(E_1) \cup \Lim_X(E_2)$. WLOG, let $x \in \Lim_X(E_1)$. Then $x \in X$ and $E_1 \setminus \{x \}$ intersects with any open neighborhood of $x$, say, $W$. So we can say that there is some $\ell \in W \cap (E_1 \setminus \{x \})$. Thus $\ell \in W$ and $\ell \in E_1 \setminus \{x \}$ (i.e. $\ell \in E_1 $ and $\ell \neq x$). Trivially, $\ell \in E_1 \cup E_2$. So then $\ell \in (E_1 \cup E_2) \setminus \{x \}$ as $\ell \neq x$. As $\ell \in W$ as well, then $\ell \in W \cap ( (E_1 \cup E_2) \setminus \{x \})$. Therefore $x \in \Lim_X(E_1 \cup E_2)$.
\end{proof}
\end{tcolorbox}

\begin{tcolorbox}[colback=black!5!white,colframe=black!75!black,title= Chapter 4: Exercise 1.2.] Let $X$ be a metric space, and assume $E \subseteq Y \subseteq X$. Prove that 
\[
\Lim_Y (E) = \Lim_X(E) \cap Y
\]
\tcblower
\begin{proof} $(\subseteq)$ Let $p \in \Lim_Y(E)$. Then $p \in Y$ and for any open neighborhood $p \in V \subseteq Y$, we have that $V \cap (E\setminus \{ p \})$ intersect. As $Y \subseteq X$, then $p$ has an open neighborhood $V \subseteq X$, and so $p \in \Lim_X(E)$ since $V \cap (E \setminus \{ p \})$ intersect. Thus $p \in Y$ and $p \in \Lim_X(E)$, and so $p \in \Lim_X(E) \cap Y$.

$(\supseteq)$ Suppose that $p \in \Lim_X(E)$ and $p \in Y$. Then $p \in X$ and for any open neighborhood $p \in V\subseteq X$, we have that $V \cap (E \setminus \{p \}) $ intersect. Now consider $W = V \cap Y$. Clearly $p \in W$, and by Theorem 2.13, $W$ is open in $Y$ since $W \subseteq Y \subseteq X$ and $V$ is an open set of $X$. Thus $p \in W$ is an open neighborhood with respect to $Y$. Suppose $\ell \in V \cap (E\setminus \{ p \})$. Then $\ell \in V $ and $\ell \in E \setminus \{ p \}$. But as $V \subseteq W$, then $\ell \in W$, so $\ell \in W \cap (E\setminus \{ p \})$. That is, $W$ and $E\setminus \{ p \}$ interesect. Thus $p \in \Lim_Y(E)$.
\end{proof}
\end{tcolorbox}

\begin{tcolorbox}[colback=black!5!white,colframe=black!75!black,title= Chapter 4: Exercise 1.3.] If $(X, \mathcal T)$ is a topological space and $E$ is a subset of $X$, we say that $x$ is a limit point of $E$ with respect to $X$ if every neighborhood of $x$ in $X$ (that is, every $U \in \mathcal T$ such that $x \in U$) intersects $E \setminus \{ x \}$.
\begin{itemize}
	\item [(a)] Suppose $\mathcal B$ is a basis for a topology $\mathcal T$ on $X$. Show that $x$ is a limit point of $E$ with respect to $X$ if and only if every $B \in \mathcal B$ containing $x$ intersects $E \setminus \{x \}$.
	\item [(b)] Show that if $E$ is any subset of $\rr$ which is not bounded above (with respect to the usual order relation on $\rr$), then $+\infty$ is a limit point of $E$ with respect to $\overline{\rr}$ (in its standard topology).
\end{itemize}
\tcblower
\begin{proof} Suppose $(X, \mathcal T)$ is a topological space, and $E$ is a subset of $X$.

(a) Suppose $\mathcal B$ is basis for a topology $\mathcal T$, i.e. every element in $\mathcal T$ can be written as union of elements of $\mathcal B \subseteq \mathcal T$. 

$(\Rightarrow)$ Assume that $x$ is a limit point of $E$ with respect to $X$. Then $x \in X$ and for any open neighborhood $x \in W \subseteq X$, we have that $W \cap (E \setminus  \{x \} )\neq \varnothing$. As $W$ is open, then it can we written as a union of basis open sets of $\mathcal B$, say, $W = \bigcup_i B_i $, where $B_i \in \mathcal B$. Thus $(\bigcup_i  B_i )\cap (E \setminus \{x \})\neq \varnothing$, so take $\ell$ in the intersection. Then $\ell \in \bigcup_i B_i$ and $\ell \in E \setminus \{ x\}$. So $\ell \in \bigcup _i B_i$ implies that $\ell \in B_\alpha$ for some $B_\alpha \in \mathcal B$. Hence $\ell \in B_\alpha \cap (E\setminus \{x \})$. Therefore the forward direction claim follows.  

$(\Leftarrow)$ Assume that for every $x \in B \in \mathcal B$ intersects $E \setminus \{x \}$. Then, in any case, we can construct an open set $L = \bigcup _j B_j$ where each $B_j \in \mathcal B$, as $\mathcal B$ is a basis. Now, as every $B_\beta \cap (E\setminus \{ x\} ) \neq \varnothing$ by hypothesis, where $B_\beta \in \mathcal B$, we have some element, say, $\ell$ in the intersection. So then $\ell \in B_\beta$ and $\ell \in E \setminus \{x \}$. Hence $\ell \in L = \bigcup _j B_j$ as $\ell \in B_\beta \in \mathcal B$. Therefore $\ell \in L \cap (E\setminus \{ x \})$ since $L$ is indeed an open neighborhood of $x$ as $x \in B_\beta$ which intersects with $E \setminus \{x \}$; that is, $x$ is a limit point of $E$ with respect to $X$. 

(b) Suppose that $E \subseteq \rr$, and $E$ is not bounded above. Then we must show that $+ \infty \in \Lim_{\overline{\rr}} (E)$. Firstly, $+\infty \in \overline{\rr}$, by definition, so it remains to show that for any open neighborhood $ +\infty \in U \subseteq \overline{\rr}$, we have that $U \cap E \setminus \{+\infty \}$ intersect. As $U$ is an open neighborhood of $+\infty$, then $U = (a, +\infty)$ for some $a \in \rr$. And as $E$ is not bounded above, then $E = (q, +\infty)$ or $E = [q, + \infty)$ for some $q \in \rr$. Consider $E = (q, + \infty)$, and $a >q$, then $U = (a, + \infty)$ intersects with $(q, +\infty) \setminus \{ + \infty \} = (q, \ell)$ clearly; and if $a<q$, then the conclusion is the same. Similarly, if $E = [q, + \infty)$, and WLOG $a <q$, then $E=[q, +\infty) \setminus \{ + \infty \}$ intersects with $(a, + \infty)$.
\end{proof}
\end{tcolorbox}

\begin{tcolorbox}[colback=black!5!white,colframe=black!75!black,title= Chapter 4: Exercise 1.4.]
Let $(X,d)$ be a metric space, and assume $Y \subseteq X$. Let $(x_n)^\infty_{n=1}$ be a sequence in $Y$ and let $x$ be a point of $X$. Prove that the following two statements are equivalent:
\begin{itemize}
	\item [(1)] $x_n \to x$ in $X$, and $x \in Y$.
	\item [(2)] $x_n \to x$ in $Y$.
\end{itemize}
\tcblower 
\begin{proof}
	$(1) \Rightarrow (2)$ Suppose that $x_n \to x$ in $X$ and  $x \in Y$. Then, as $(x_n)_{n=1}^\infty$ is a sequence in $Y$ and $x$ is a point of $X$, and we assume $x_n \to x$ in $X$, then for every open neighborhood $x \in V \subseteq X$ (so $x \in V \cap Y \subseteq X$) there is some $N \in \nn$ such that $n \geq N$ implies that $x_n \in V$ (so $x_n \in V \cap Y$). But as $x \in Y$, then $x \in Y \cap V \subseteq Y$ so $V \cap Y$ is open in $Y$ by Theorem 2.13 and so $x_n \in V \cap Y \subseteq Y$. Thus $x_n \to x$ in $Y$.
	
	$(2) \Rightarrow (1)$ Suppose $x_n \to x$ in $Y$. Then every open neighborhood $x \in W \subseteq Y$ there exists an $N \in \nn$ such that $n \geq N$ implies $x_n \in W$. So $x \in Y$. But as $W \subseteq Y \subseteq X$, then $W$ is open in $Y$ if and only $W = E \cap Y$ for some open set $E$ of $X$. So then we have some $E$ open in $X$ such that $x \in W = E \cap Y \subseteq X$ is an open neighborhood of $x$; thus $x \in E$ and $x_n \in E$ as $W \subseteq E$, an open neighborhood of $x$, and $x_n \to x$ in $X$.
\end{proof}
\end{tcolorbox}


\end{document}